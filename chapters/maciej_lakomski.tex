\documentclass{article}
\usepackage{graphicx}
\graphicspath{ {./pictures/} }
\title{Pixel}
\author{Maciej Łakomski}
\date{October 2023}

\begin{document}


\maketitle

\section{Introduction}
\[ (p+i)^2+1/37 = x/e-l \]
\begin{center}
\includegraphics{ml.jpg}
  \label{ml}
\end{center}


\begin{table}
    \centering
    \begin{tabular}{ccccc}
        1 & 2 & 3 & 4 & \\
        5 & 6 & 7 & 8 & \\
        9 & a & b & c & \\
    \end{tabular}
    \label{tab:my_label}
\begin{itemize}
  \item cos na liscie1
  \item cos na liscie2
  \item cos na liscie3
\end{itemize}
\begin{enumerate}
  \item cos na liscie4
  \item cos na liscie5
  \item cos na liscie6
\end{enumerate}
\begin{center}
    Pierwszy akapit \textbf{ala} ma kota

    Drugi akapit ala \underline{ma} kota
\end{center}
    
\end{table}


\ref{ml}
\pageref{ml}


\end{document}
