\documentclass{article}
\usepackage{graphicx}
\usepackage{lipsum}

\begin{document}

\section{Wprowadzenie}
W tym dokumencie przedstawimy przykłady wykorzystania różnych elementów w LaTeX, takich jak wyrażenie matematyczne, zdjęcie, tabela, listy, tekst z formatowaniem oraz odwołania do figur i tabel.

\section{Wyrażenie Matematyczne}
Oto przykład wyrażenia matematycznego w równaniu:
\[
E=mc^2
\]

\section{Zdjęcie}
Poniżej przedstawiamy przykład wstawienia zdjęcia. Na Rysunku \ref{fig:example} widzimy przykład.

\begin{figure}[h]
    \centering
    \includegraphics[width=0.5\textwidth]{example-image.jpg}
    \caption{Przykładowe zdjęcie}
    \label{fig:example}
\end{figure}

\section{Tabela}
Teraz pokażemy przykład tabeli. Tabela \ref{tab:example} przedstawia dane.

\begin{table}[h]
    \centering
    \begin{tabular}{|c|c|}
        \hline
        Kolumna 1 & Kolumna 2 \\
        \hline
        Wiersz 1 & Wartość 1 \\
        Wiersz 2 & Wartość 2 \\
        \hline
    \end{tabular}
    \caption{Przykładowa tabela}
    \label{tab:example}
\end{table}

\section{Listy}
Poniżej znajduje się przykład listy numerowanej:
\begin{enumerate}
    \item Pierwszy item
    \item Drugi item
    \item Trzeci item
\end{enumerate}

A teraz lista nienumerowana:
\begin{itemize}
    \item Pierwszy item
    \item Drugi item
    \item Trzeci item
\end{itemize}

\section{Tekst z Formatowaniem}
Oto przykład tekstu z formatowaniem. Lorem ipsum dolor sit amet, consectetur adipiscing elit. Sed non vulputate nunc. In hac habitasse platea dictumst.

\subsection{Podrozdział}
Nulla facilisi. Morbi nec magna eget dui sodales rhoncus. Fusce in libero et est malesuada interdum. 

\section{Odwołania do Figur i Tabel}
Jak możemy zobaczyć na Rysunku \ref{fig:example} oraz w Tabeli \ref{tab:example}, ...

\end{document}